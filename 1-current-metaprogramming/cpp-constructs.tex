\documentclass[../main]{subfiles}
\begin{document}

\section{Language constructs of \cpp metaprogramming}

In this section I will cover the constructs of \cpp metaprogramming.
Metaprograms are defined as programs that take code as input and/or output,
therefore I will categorize metaprogramming constructs as:
those used for input and output,
and those that enable compile time computing.

This distinction will be important to explain the work presented in
this thesis, as it focuses a lot on methods to translate
compile time compute results into \cpp programs.

\subsection{Input and output of metaprograms}

\subsubsection{Templates}

Templates are the cornerstone of the \cpp metaprogramming model.
They are the main \cpp mechanism to generate structures, functions,
or even variables from types or values known at compile time.
They were originally meant to be used for implementing generic types,
but over time the extension of their features made them Turing complete.

\paragraph{Different kinds of templates}

\cpp templates offer ways to output code for both data structures
and functions.

\begin{itemize}
\item Type templates

\begin{lstlisting}[
  language=c++,
  caption=Type template example,
  label=lst:type-template-example
]{}
/// Generic anotated type
template <typename T> struct named_value_t {
  std::string name;
  T value;
};
\end{lstlisting}

\begin{lstlisting}[
  language=c++,
  caption=Type alias template example,
  label=lst:type-alias-template-example
]{}
/// Alias template for nested named value
template <typename T>
using nested_named_value_t =
    named_value_t<named_value_t<T>>;
\end{lstlisting}

Templates can be used with type aliases too, as shown in
listing \ref{lst:type-alias-template-example}.

\item Function templates

\begin{lstlisting}[
  language=c++,
  caption=Function template example,
  label=lst:function-template-example
]{}
/// Returns a value annotated with
/// its string representation
template <typename T>
named_value_t<T> make_named_value(T value) {
  return named_value_t<T>(std::to_string(value),
                          std::move(value));
}
\end{lstlisting}

\item Variable templates

\begin{lstlisting}[
  language=c++,
  caption=Variable template example,
  label=lst:variable-template-example
]{}
/// Returns the default value of T annotated with
/// its string representation
template <typename T>
named_value_t<T> annotated_default_v =
    make_named_value(T{});

named_value_t<int> val = annotated_default_v<int>;
\end{lstlisting}

\end{itemize}

\paragraph{Different kinds of parameters}

\begin{itemize}
\item Type parameters
\item Non-type template parameters (\nttp)
\item Parameter packs
\end{itemize}

\paragraph{Advanced \cpp template mechanisms}

\begin{itemize}
\item Substitution failure is not an error (\sfinae)
\item Template specialization
\end{itemize}

\subsubsection{Overloading}

\subsection{Compile time logic}

\subsubsection{\Tmp and constant evaluation}

\begin{itemize}
\item Computation using types
\item Computation using values
\end{itemize}

\subsubsection{Constexpr functions}

\begin{itemize}
\item \constexpr functions
\item \constexpr memory allocation
\end{itemize}

The use of \constexpr functions for compile time programming is preferable
for many reasons: familiar to most \cpp developers (and therefore more
maintainable), it allows the use of types to enforce semantics properly
as opposed to type-based metaprogramming, and it scales much better than \tmp.

% Templates allow types, variables, and functions to be generated from
% compile time parameters. These parameters used to be only types,
% but over time \cpp evolved to allow developers to use typed values
% and parameter packs.
%
% Function overloading allows several functions to bear the same name
% as long as they cannot be invoked with parameters of the same type.
% This allows selection of code paths to happen at compile time.
%
% Although these two mechanisms are enough to build very capable
%
% \subsection{
%   Templates
% }
%
% \subsubsection{
%   Different kinds of templates
% }
%
% \paragraph{
%   Types
% }
%
% An example:
%
% \begin{lstlisting}[
%   language=c++,
%   caption=Some example
% ]
% template<typename T> struct toto_t {};
% \end{lstlisting}
%
% \paragraph{
%   Functions
% }
%
% An example:
%
% \begin{lstlisting}[
%   language=c++,
%   caption=Some example
% ]
% template<typename T> struct toto_t {};
% \end{lstlisting}
%
% \paragraph{
%   Variables
% }
%
% An example:
%
% \begin{lstlisting}[
%   language=c++,
%   caption=Some example
% ]
% template<typename T> struct toto_t {};
% \end{lstlisting}
%
%
% \subsubsection{
%   Different kinds of parameters
% }
%
% ...
%
% \subsection{Overloading}
%
% \subsection{Constant evaluation}
%
% \subsection{\lstinline{if constexpr}}
%
% \subsection{Parameter packs}
%
% \subsection{Constexpr functions and variables}


\end{document}
