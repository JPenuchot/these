\documentclass[../../../main]{subfiles}
\begin{document}

\section{Experiments and results}

In this section, we will use the techniques we already mentioned to implement
embedded compilers for the Brainfuck language,
as well as a subset of \LaTeX math.

The Brainfuck embedded compiler will help us evaluate code generation from
a pointer tree structures. We will be parsing Brainfuck programs from strings
to an AST, which we will then convert into \cpp code using the techniques
introduced in \ref{lbl:ptr-tree-codegen}.

On the other hand the \LaTeX math parser will illustrate a use case where
the output of a parsing algorithm can be used almost directly
for code generation. Moreover, we will show how this can be leveraged
to generate high performance math kernels
using the Blaze library\cite{blazelib}.

These two experiments will be used as benchmarks to measure
the impact of embedded compilation on overall \cpp compilation times,
which will help us establish guidelines for the implementation of
embedded compilers in \cpp23.
We will then discuss the issues that emerge from the limitations set by the
\cpp standard, as well as modifications that could be made to make
\constexpr based metaprogramming easier
and with less impact on compilation times.

\subsection{Brainfuck parsing: Generating code from an AST}

Brainfuck is a structured imperative language
characterized by 8 tokens that allow increasing or decreasing a pointer
to single byte values in a fixed size array, acting on the pointee value by
increasing or decreasing it, single character input/output, and looping
as long as the pointee value is non-zero. Every instruction
can be directly translated into \cpp code as shown in table \ref{fig:BF}.

\begin{figure}
\begin{tabular}{|l|l|}
\hline
\BF token & \cpp equivalent \\ \hline
(Program start) & \lstinline|char array[30000] = {0}; char *ptr = array;| \\
\lstinline|>| & \lstinline|++ptr;| \\
\lstinline|<| & \lstinline|--ptr;| \\
\lstinline|+| & \lstinline|++*ptr;| \\
\lstinline|-| & \lstinline|--*ptr;| \\
\lstinline|.| & \lstinline|std::putchar(*ptr);| \\
\lstinline|,| & \lstinline|*ptr = std::getchar();| \\
\lstinline|[| & \lstinline|while (*ptr) {| \\
\lstinline|]| & \lstinline|}| \\
% Any other token & Nothing \\
\hline
\end{tabular}
\caption{
  \BF token to \cpp code conversion table --
  https://en.wikipedia.org/wiki/Brainfuck
}
\label{fig:BF}
\end{figure}

The Brainfuck language will be a good candidate for studying
compilation time scaling. The size of the AST is easily predictable as
each character is converted directly into an AST node,
as shown in figure \ref{fig:BF}.

\subsubsection{Constexpr Brainfuck parsing}

\paragraph{A Brainfuck AST}

The Brainfuck AST is very similar to the pointer tree structure defined in
\ref{lst:tree_generation}. The major difference is the use of inheritance
polymorphism to differentiate the different kinds of tokens.

\paragraph{A \constexpr Brainfuck parser}

\paragraph{Performance comparison of the 3 AST conversion techniques}

\subsubsection{Experiments}

\paragraph{Variable sized benchmarks}

% Show the kind of code we generate

% Graphs

\paragraph{"Real" programs}

% Hello World
% Mandelbrot

% Performance table

\subsubsection{Discussion}

% ...

\subsection{Parsing to a serialized representation}

% We want a more realistic/useful use case: parsing math forumulae
% for high performance code generation

% Goal of the experiment: demonstrate interoperability and study the performance
% impact of using vs not using compile time parsing in the context of high
% high performance code generation.

% The fact that we're using Blaze makes the translation unit (explain)
% much larger and may have an impact on meta-compilation performance

% First: What algorithm

\subsubsection{The Shunting-Yard algorithm}

% Introduce the Shunting-Yard algorithm

\subsubsection{The Blaze library}

% Introduce Blaze

% Say we translate the Shunting-Yard result to Blaze expression templates

\subsubsection{Experiment results}

% ...

\end{document}
