\documentclass[../main]{subfiles}
\begin{document}

% \begin{abstract}
% Design of high performance, high abstraction libraries in \cpp often take
% advantages of techniques like template metaprogramming or expression templates
% to design \dsels. However, such techniques are limited by the natural syntax
% of \cpp. In this paper, we explore the benefits of using \cpp23 compile time
% computation features to provide compile time string based \dsels that can then
% use arbitrary syntax.
% \end{abstract}

\section{Introduction}

\cpp is often touted as a \textit{Zero-Cost Abstraction} langage due to some of
its design philosophy and its ability to compile abstraction to a very efficient
binary code. Some more radical techniques can be used to force the compiler to
interpret \cpp code as a \dsel. Template metaprogramming is such a technique
and it spawned a large corpus of related idioms from compile time function
evaluation to lazy evaluation via \textit{Expression Templates}.
\\

In the field of High Performance Computing, \cpp users are often driven to use
libraries built on top of those idioms like Eigen\cite{eigen} or
Blaze\cite{blazelib,iglberger2012_2}. They all suffer from a major limitation:
by being tied to the natural \cpp syntax, they can't express nor embed arbitrary
languages.
\\

In this paper, we try to demonstrate that the new features of \cpp23 related to
compile time programming are able to help developers designing \dsels with
arbitrary syntax by leveraging \constexpr computations, compile time dynamic
objects and lazy evaluation through lambda functions. After contextualizing our
contribution in the general \dsels domain, this paper will explain the core
new techniques enabled by \cpp23 and how we can apply to build two different
\dsels with their own non-\cpp syntax. We'll also explore the performances of
said \dsels in term of compile time to assess their usability in realistic code.

\subfile{1-technical-background/main.tex}
\subfile{2-constexpr-codegen-techniques/main.tex}
\subfile{3-experiments/main.tex}
\subfile{4-evaluation/main.tex}

\section{Conclusion}

% [TODO] Talk about how pass-by-generator actually kinda sucks
% because of constexpr constraints, and how serializing takes more code
% but less time debugging stuff related to constexpr constraints.
% Serializing makes dealing with dynamic memory easier because codegen functions
% only deal with static memory.

We wanted to demonstrate that using \constexpr code to implement parsers for
\dsel of arbitrary syntax in \cpp23 is possible despite limitations on
\constexpr memory allocation, and that doing so is possible with reasonable
impact on compilation times.
\\

We achieved that by implementing a \constexpr parser for the Brainfuck language,
with code generation backends implementing three different strategies to
transform \constexpr program representations into code using function
generators, expression templates, and non-type template parameters.
We also demonstrated the interoperability of these \constexpr parsers by
implementing a parser for mathematical languages that can be used as a frontend
for existing high performance \cpp computation libraries.
\\

Our benchmarks highlight compilation time scaling issues with pass-by-generator
and expression template code generation strategies for large programs, and
excellent scaling capabilities for non-type template parameter based code
generation strategies. These results can be used to decide which strategy to
adopt for the implementation of future \dsel based on \constexpr parsers
based on considerations for compilation times or implementation complexity.
\\

Going forward, \constexpr parser generators could help reduce
\dsel implementation time and help embed more languages into \cpp23.
Further research has to be made to determine the impact of such generators on
\dsel implementation complexity and compilation times.

% On avait tel objectif X et Y
% On montre qu'on a fait X et que ca permet Y
% Les benchmarks montrent la pluvalue avec un temps de compil raisonnable
% Nos travaux futurs: AI deep learning blockchain

% TODO: Talk about Cest

\subfile{appendix/main.tex}

\end{document}
