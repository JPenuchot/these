\documentclass[main]{subfiles}
\begin{document}

Depuis le d\'ebut de l'informatique les architectures des ordinateurs n'ont
cess\'e de croitre en complexit\'e. Le nombre de transistor dans ces derniers
double encore aujourd'hui tous les 3 ans, et la fin de ce doublement
quasi-exponentiel n'est pour le moment pas en vue.
En revanche, la capacit\'e de traitement s\'equencielle a connu un brusque
ralentissement avec la fin de l'augmentation des fr\'equences des processeurs:
on parle de la fin du <<free lunch>>.

La conception des processeurs a donc \'evolu\'e vers des architectures
parall\`eles, qui permettent de d\'ecouper des traitements en sous-t\^aches
pour les faire ex\'ecuter simultan\'ement par plusieurs unit\'es de traitement.
La nature et la granularit\'e du d\'ecoupage peut varier selon les
architectures:

\begin{itemize}
\item
\textbf{Le mod\`ele SIMD} consiste \`a avoir, au sein d'un c\oe{}ur de
processeur, des registres vectoriels permettant d'ex\'ecuter
\textbf{une instruction} sur plusieurs valeurs dans un seul registre,
de mani\`ere simultan\'ee.

\item
\textbf{Le mod\`ele multi-c\oe{}ur} permet \`a plusieurs c\oe{}urs ind\'ependants de
fonctionner en parall\`ele. Ces c\oe{}urs partagent la m\^eme m\'emoire dans le
cas des architectures \`a m\'emoire unifi\'ee, mais il est \'egalement possible
d'avoir plusieurs processeurs ind\'ependants ne partageant pas de m\'emoire
et n\'ecessitant donc d'effectuer des transferts de l'un \`a l'autre dans le cas
des architectures \`a m\'emoire non unifi\'ee.

\item
\textbf{Les processeurs graphiques (GPU)} permettent d'ex\'ecuter des <<warps>> en
parall\`ele, \`a condition que ceux-ci soient toujours synchrones. Dans le cas
o\`u des warps se d\'esynchronisent, ces derniers doivent s'attendre avant de
poursuivre leur ex\'ecution parall\`ele. Ce mod\`ele est tr\`es proche
du mod\`ele SIMD.
\end{itemize}

De nombreuses mani\`eres d'ex\'ecuter du code en parall\`ele existent,
et peuvent co-habiter au sein d'un m\^eme ordinateur, ce dernier pouvant faire
partie d'un cluster.

La diversit\'e des architectures n\'ecessite de nouvelles adaptations pour que
les applications dites <<haute performance>> puissent s'ex\'ecuter
le plus rapidement possible sur le plus d'architectures possibles.
De nouvelles techniques de programmation ont vu le jour, permettant d'\'ecrire
des programmes dont le code peut \^etre sp\'ecialis\'e lors de la compilation
en fonction de donn\'ees connues lors de la compilation;
c'est le principe de l'\'evaluation partielle. D'autres techniques permettent
de manipuler des programmes en entr\'ee comme en sortie lors de la compilation,
nous parlerons dans ce cas de \textbf{m\'etaprogrammation}.

La m\'etaprogrammation permet donc de compiler des programmes en ex\'ecutables
optimis\'es pour des architectures cibl\'ees. On retrouve ces techniques
dans de nombreux langages, l'exemple historique le plus notable \'etant
Lisp, paru en 1960, dont le syst\`eme de macros permet d'utiliser
l'int\'egralit\'e du langage Lisp pour g\'en\'erer des programmes Lisp.
Le langage C, lui, permet d'injecter des tokens lors de la phase de
preprocessing qui pr\'ec\`ede l'analyse syntaxique du code. En \cpp,
des techniques de m\'etaprogrammation plus riches bas\'ees sur son syst\`eme de
\textbf{templates} permet de d\'efinir des structures et fonctions
param\'etriques, dont les param\`etres (des types ou des valeurs) sont
connus \`a la compilation. Le langage \'evoluera par la suite pour permettre
d'\'evaluer du code \cpp \`a la compilation pour construire des param\`etres de
templates.

Ces techniques, en plus des aspects <<haute performance>> du langage \cpp,
ont fait na\^itre un \'ecosyst\`eme de biblioth\`eques m\'etaprogramm\'ees
permettant d'optimiser automatiquement des programmes pour des architectures
tr\`es diverses: les architectures SIMD, les processeurs multi-coeurs, les GPU,
ou tout autre type d'acc\'el\'erateur programmable.

Certaines de ces biblioth\`eques utilisent la m\'etaprogrammation pour proposer
des langages d\'edi\'es, reposant entre autres sur la surcharge
d'op\'erateurs pour impl\'ementer des langages bas\'es sur la syntaxe \cpp.
D'autres, comme Compile Time Regular Expressions, utilisent
la m\'etaprogrammation de templates pour impl\'ementer des analyseurs
syntaxiques permettant de reconna\^itre des langages arbitraires.

Depuis \cpp20, l'allocation dynamique peut \^etre utilis\'ee dans des fonctions
ex\'ecut\'ees \`a la compilation d'un programme \cpp. Nous souhaitons \'etudier
l'impact de cette nouvelle fonctionnalit\'e sur l'impl\'ementation d'analyseurs
syntaxiques de langages arbitraires, autant sur la conception des analyseurs
syntaxiques que sur leur performance en temps de compilation.
Comment peut-on exploiter la m\'emoire dynamique dans l'\'evaluation de code
\cpp \`a la compilation pour l'int\'egration de langages d\'edi\'es de syntaxes
arbitraires pour la programmation <<haute performance>> ?

\section{
  G\'en\'eration de code SIMD performant
}

Pour r\'epondre \`a ces questions, nous commencerons d'abord par montrer
l'efficacit\'e de la m\'etaprogrammation \cpp pour la programmation
<<haute performance>> \`a travers l'impl\'ementation d'une version
g\'en\'erique d'une routine d'alg\`ebre lin\'eaire tr\`es r\'epandue:
la multiplication matrice-vecteur, avec une matrice en disposition
<<column-major>>.

L'impl\'ementation repose sur une couche d'abstraction pour les registres et
instructions SIMD permettant de conna\^itre leur taille \`a la compilation.
En exploitant cette donn\'ee, nous pouvons g\'en\'erer des noyaux de calcul
s'adaptant \`a la taille de ces derniers pour assurer un niveau de performance
tr\`es \'elev\'e pour une multitude de jeux d'instructions SIMD: SSE4.2 et AVX2
pour x86, et NEON pour ARM.

\begin{figure}[h]
\fontsize{8}{10}\selectfont
\includesvg[width=\textwidth]{1-current-metaprogramming/images/gemv-fig2}
\caption{
  Performance de GEMV sur processeur Intel i5-7200
  avec le jeu d'instructions SSE4.2
}
\label{fig:gemv-sse-bench-fr}
\end{figure}

\begin{figure}[h]
\fontsize{8}{10}\selectfont
\includesvg[width=\textwidth]{1-current-metaprogramming/images/gemv-fig3}
\caption{
  Performance de GEMV sur processeur Intel i5-7200
  avec le jeu d'instructions AVX2
}
\label{fig:gemv-avx-bench-fr}
\end{figure}

\begin{figure}[h]
\fontsize{8}{10}\selectfont
\includesvg[width=\textwidth]{1-current-metaprogramming/images/gemv-fig4}
\caption{
  Performance de GEMV sur processeur ARM Cortex A57
}
\label{fig:gemv-arm-bench-fr}
\end{figure}

Nous comparons ensuite les performances du code g\'en\'er\'e avec des
impl\'ementations en assembleur d\'eclin\'ees pour chaque micro-architecture
fournies par la biblioth\`eque OpenBLAS en l'appliquant sur des matrices de
tailles allant de $4 \times 4$ \`a $512 \times 512$.
Malgr\'e les r\'esultats peu concluants en SSE4.2
(figure \ref{fig:gemv-sse-bench-fr}), les r\'esultats se montrent
tr\`es positifs en AVX2 (figure \ref{fig:gemv-avx-bench-fr}) o\`u nous obtenons
des r\'esultats sup\'erieurs ou \'equivalents \`a OpenBLAS. Quant \`a ARM
(figure \ref{fig:gemv-arm-bench-fr}), les r\'esultats nous y obtenons
des performances sup\'erieures \`a celles d'OpenBLAS quelle que soit la taille
de la matrice.
\\

Ce chapitre a fait l'objet d'une publication scientifique \`a la conf\'erence
internationale \textbf{High Performance Computation \& Simulation 2018}.

\section{
  Nouvelle m\'ethode pour l'analyse des temps de compilation
}

Nous pr\'esentons ensuite une nouvelle m\'ethode pour l'analyse des temps
de compilation de m\'etaprogrammes \cpp, ainsi que des outils impl\'ementant
cette m\'ethode de mani\`ere reproductible. Cette m\'ethode
repose sur l'utilisation des donn\'ees de profiling de Clang pour permettre
des analyses cibl\'ees, et permet de r\'ealiser des \'etudes de temps
de compilation sur des m\'etaprogrammes param\'etrables, et donc d'observer
les temps de compilation en fonction de la taille de ces derniers.

Le projet d'outillage \textbf{ctbench} qui l'accompagne repose sur CMake
comme langage d'interface principal car il est tr\`es largement utilis\'e
au sein de la communaut\'e \cpp. Il permet de g\'en\'erer des graphes dans
diff\'erents formats, y compris vectoriels, avec un syst\`eme de param\'etrage
lui permettant de cibler et grouper des cat\'egories d'\'ev\'enements
dans le processus de compilation.
Le moteur de g\'en\'eration des graphes est modulaire
et permet l'ajout de nouveaux types de graphes via une interface
de programmation clairement document\'ee. Il s'accompagne d'une biblioth\`eque
\cpp pour exploiter les hi\'erarchies de fichiers g\'en\'er\'ees
par \textbf{ctbench}.

Cette m\'ethode permet de mieux comprendre l'impact des
techniques de m\'etaprogrammation sur les temps de compilation en permettant
d'observer finement chaque \'etape de la compilation \'etant donn\'e que les
mesures du profiler de Clang sont accompagn\'ees de m\'etadonn\'ees relatives
aux symboles \cpp trait\'es lors de la compilation.
\\

Ce chapitre a fait l'objet d'une publication scientifique au
\textbf{Journal of Open Source Software}, et d'une pr\'esentation
\`a \textbf{CPPP 2021}.

\section{
  Des nouvelles m\'ethodes pour la g\'en\'eration de code
}

\cpp20 a rendu possible l'allocation de m\'emoire dynamique dans les programmes
\cpp ex\'ecut\'es \`a la compilation. Toutefois, la m\'emoire allou\'ee
ne peut \^etre utilis\'ee directement comme param\`etre de template et
il faut donc trouver des solutions pour exploiter son contenu autrement.

Une premi\`ere solution consiste \`a passer, pour chaque \'el\'ement contenu
dans une structure dynamique, une fonction qui renvoie ledit \'el\'ement.
Cette option a un co\^ut car les r\'esultats interm\'ediaires ne peuvent pas
\^etre stock\'es, ainsi la fonction g\'en\'erant la structure devra \^etre
rappel\'ee pour chaque \'el\'ement de la structure. Cette mani\`ere de passer
les \'el\'ements d'une structure arbitraire en param\`etre de template est
baptis\'ee <<pass-by-generator>> (ou <<passage par g\'en\'eratrice>>),
et permet de g\'en\'erer du code \`a partir
d'une structure dynamique sans n\'ecessiter de repr\'esentation interm\'ediaire.
Elle peut toutefois \^etre utilis\'ee pour g\'en\'erer une repr\'esentation
interm\'ediaire sous forme d'arborescence de templates de types (appel\'ees
<<expression templates>>), ce qui est commun dans les bibiloth\`eques de
m\'etaprogrammation \cpp.

Une autre solution consiste \`a s\'erialiser une structure dynamique pour
la convertir en tableau de taille fixe, en rempla\c{c}ant les pointeurs vers
les sous-\'el\'ements par des indices pointant \`a l'int\'erieur du tableau
dans lequel la structure est s\'erialis\'ee. De cette mani\`ere, la structure
peut \^etre transform\'ee en tableau de taille fixe ne contenant aucun pointeur
vers de la m\'emoire dynamique allou\'ee \`a la compilation, et ce tableau
peut \^etre utilis\'e directement en param\`etre de template pour parcourir
la structure et g\'en\'erer du code.
Cette technique n\'ecessite plus d'effort, mais on peut s'attendre \`a
ce qu'elle soit moins co\^uteuse en temps de compilation car elle ne n\'ecessite
d'appeler la fonction de g\'en\'eration de la structure que deux fois.

\section{
  \'Etude de cas: int\'egration du langage Brainfuck dans \cpp
}

Pour \'etudier ces nouvelles techniques de m\'etaprogrammation,
nous les appliquons pour l'impl\'ementation d'un analyseur syntaxique et
de g\'en\'erateurs de code pour le langage Brainfuck.
L'impl\'ementation du parseur est triviale, et sa sortie est un arbre syntaxique
sous forme d'arbres de pointeurs, similaire \`a ce qui est utilis\'e dans les
compilateurs d\'evelopp\'es en \cpp.
Les g\'en\'erateurs de code utilisent les techniques pr\'esent\'ees
dans le chapitre pr\'ec\'edent pour transformer l'AST en code \cpp.

Nous effectuons ensuite des benchmarks pour comparer
les diff\'erentes solutions: des benchmarks bas\'es sur
des programmes existants, l'un \'etant un Hello World! et l'autre affichant
la fractale de Mandelbrot, ainsi que des benchmarks synth\'etiques effectu\'es
\`a l'aide de \textbf{ctbench} pour \'etudier le passage \`a l'\'echelle
des diff\'erentes techniques plus en finesse.

Lors de ces benchmarks, nous constatons que la m\'ethode dite
<<pass-by-generator>> induit des temps de compilation quadratiques en raison
du nombre d'appels \`a la fonction de parsing, appell\'ee elle-m\^eme par les
fonctions g\'en\'eratrices pour chaque n\oe{}ud de l'arbre de syntaxe.
Cette technique est suffisante pour des programmes de petite taille,
mais sa complexit\'e quadratique la rend inutilisable
sur des cas de grande taille.

La m\'ethode consistant \`a s\'erialiser l'arbre, quant \`a elle, n'appelle
la fonction de parsing que deux fois. De fait, sa complexit\'e est lin\'eaire
et permet de compiler l'exemple affichant la fractale de Mandelbrot qui compte
environ 11000 n\oe{}uds en moins de 20 secondes (voir figure
\ref{fig:bf-compile-times-fr}).

\begin{figure}[h]
\begin{tabular}{|c|c|c|c|}
\hline
Backend           & Hello World & Hello World x2  & Mandelbrot \\
\hline
Flat (Monolithique) & 0.63        & 0.80            & 18.16 \\
Flat (Surcharg\'e)  & 0.66        & 0.90            & 28.51 \\
\gls{pbg}           & 3.55        & 12.73           & \'Echec (timeout) \\
\gls{et}            & 19.18       & 74.51           & \'Echec (timeout) \\
\hline
\end{tabular}
\caption{Mesures de temps de compilation pour Brainfuck}
\label{fig:bf-compile-times-fr}
\end{figure}

Nous en concluons donc que la s\'erialisation d'un AST permet d'obtenir des
r\'esultats satisfaisants pour des cas de grande taille.
\\

Ce chapitre a fait l'objet d'une pr\'esentation \`a \textbf{Meeting C++ 2022}
dans le cadre d'un projet commun de recherche avec Paul Keir et Andrew Gozillon
de University of the West of Scotland.

\section{
  Application: int\'egration d'un langage math\'ematique dans \cpp
}

Les conclusions tir\'ees du cas de Brainfuck nous permettent d'impl\'ementer
un autre analyseur syntaxique pour un langage math\'ematique de d\'emonstration
que nous appelons Tiny Math Language (TML).
Nous d\'ecidons d'impl\'ementer un parseur bas\'e sur
l'algorithme Shunting Yard d'Edsger Dijkstra dont la sortie est la formule
en notation polonaise inverse. Cette repr\'esentation permet ensuite de
g\'en\'erer le programme correspondant \`a l'aide d'un m\'etaprogramme template
stockant les r\'esultats interm\'ediaires dans une pile sous forme de tuple.

L'analyseur syntaxique est facilement param\'etrable: il permet de d\'efinir des
fonctions, op\'erateurs (avec leur associativit\'e et leur pr\'ecedence),
des parenth\`eses, et des variables.

Nous d\'ecidons ensuite d'utiliser le parseur et le g\'en\'erateur de code
pour mesurer les temps de compilation pour des formules math\'ematiques
de 1 \`a 41 symboles. Dans un premier cas, nous l'utilisons pour effectuer une
s\'erie d'addition sur des entiers natifs.
Dans un autre cas, nous l'utilisons pour effectuer des op\'erations sur des
vecteurs de la biblioth\`eque \textbf{Blaze}. Par surcharge d'op\'erateurs,
les types de Blaze g\'en\`erent l'arborescence de formules math\'ematiques
pour ensuite g\'en\'erer le code les \'evaluant. Ce cas repr\'esente donc
l'utilisation \`a la compilation d'un parseur pour la g\'en\'eration de code dit
<<hautes performances>>.
Enfin nous introduisons un troisi\`eme cas consistant simplement \`a
g\'en\'erer les formules Blaze sans utiliser le parseur en amont.

\begin{figure}[h]
% \fontsize{8}{10}\selectfont
\includesvg[width=\textwidth]{3-new-approaches-to-metaprogramming/images/shunting-yard.addition-series}
\caption{Benchmarks de parsing et de g\'en\'eration de code}
\label{fig:tml-ctbench-fr}
\end{figure}

Dans tous les cas mesur\'es (figure \ref{fig:tml-ctbench-fr}), nous obtenons
des temps de compilation inf\'erieurs \`a 20 secondes, y compris pour des
formules de 41 symboles.
La complexit\'e  li\'ee au parsing et \`a la g\'en\'eration de code
est quadratique, mais les temps de compilation demeurent raisonnables
dans la mesure ou une formule Blaze comptant un seul symbole n\'ecessite
environ 5 secondes de temps de compilation.

\end{document}
