%%%% Modèle proposé par kira.ribeiro@universite-paris-saclay.fr  %%%%
%%%% màj : 27/01/2023 %%%%

\documentclass[english,12pt,a4paper]{book}
\usepackage[utf8]{inputenc}
\usepackage[T1]{fontenc}
\usepackage[english]{babel}
\usepackage[default,oldstyle, scale=.95]{opensans} % police Open Sans
\usepackage{amsmath}
\usepackage{amsfonts}
\usepackage{amssymb}
\usepackage{csquotes}
\usepackage{caption}
\captionsetup[figure]{font=small,labelfont=small}
\usepackage{xcolor} % ou color selon l'installation
\definecolor{Prune}{RGB}{99,0,60} % l14-33 : couleurs de la charte graphique upsaclay
\definecolor{B1}{RGB}{49,62,72}
\definecolor{C1}{RGB}{124,135,143}
\definecolor{D1}{RGB}{213,218,223}
\definecolor{A2}{RGB}{198,11,70}
\definecolor{B2}{RGB}{237,20,91}
\definecolor{C2}{RGB}{238,52,35}
\definecolor{D2}{RGB}{243,115,32}
\definecolor{A3}{RGB}{124,42,144}
\definecolor{B3}{RGB}{125,106,175}
\definecolor{C3}{RGB}{198,103,29}
\definecolor{D3}{RGB}{254,188,24}
\definecolor{A4}{RGB}{0,78,125}
\definecolor{B4}{RGB}{14,135,201}
\definecolor{C4}{RGB}{0,148,181}
\definecolor{D4}{RGB}{70,195,210}
\definecolor{A5}{RGB}{0,128,122}
\definecolor{B5}{RGB}{64,183,105}
\definecolor{C5}{RGB}{140,198,62}
\definecolor{D5}{RGB}{213,223,61}

\usepackage[absolute]{textpos}
\usepackage{array}
\usepackage{biblatex}
\usepackage{booktabs}
\usepackage{colortbl}
\usepackage{geometry}
\usepackage{glossaries}
\usepackage{graphicx}
\usepackage{hyperref}
\usepackage{listings}
\usepackage{mdframed}
\usepackage{multicol}
\usepackage{multirow}
\usepackage{subcaption}
\usepackage{svg}
\usepackage{tikz}
\usepackage{titlesec}
\usepackage{xspace}

% \renewcommand{\FrenchLabelItem}{\textbullet}

% pour ne garder que les n°de page en milieu-bas
% et supprimer les indications de chapitre en marge haute
\pagestyle{plain}

% Fichiers biblio

\addbibresource{bibliography/biblio.bib}
\addbibresource{bibliography/comptime.bib}
\addbibresource{bibliography/ctbench.bib}
\addbibresource{bibliography/hpcs2018.bib}
\addbibresource{bibliography/metalanguages.bib}
\addbibresource{bibliography/metalibraries.bib}
\addbibresource{bibliography/parallelism.bib}
\addbibresource{bibliography/poacher.bib}
\addbibresource{bibliography/simd.bib}

% Commandes pour la these de Jules

\providecommand{\cpp}{\textsc{C++}\xspace}
\providecommand{\ctbench}{\textsc{ctbench}\xspace}
\providecommand{\eg}{\textit{e.g.}\xspace}
\providecommand{\grapher}{\textsc{grapher}\xspace}
\providecommand{\ie}{\textit{i.e.}\xspace}

\hypersetup{
  colorlinks=true, % toujours garder colorlinks=true
  linkcolor=black,
  urlcolor=A5
}

\lstset{
  frame=single,
  basicstyle={\small\ttfamily},
  numberstyle=\tiny\color{C1},
  keywordstyle=\color{B2},
  commentstyle=\color{D4},
  stringstyle=\color{A3},
  breaklines=true,
  tabsize=2,
  showstringspaces=false
}

\lstdefinelanguage{cmake}{
  morekeywords={STRING, BOOL, CACHE, REQUIRED},
  morecomment=[l]{\#},
  morestring=[b]"
}

\lstdefinelanguage{rust}{
  morekeywords={let, mut, expr},
  morecomment=[l]{//},
  morestring=[b]"
}

\lstdefinelanguage{javascript}{
  morekeywords={},
  morecomment=[l]{},
  morestring=[b]"
}

\lstdefinelanguage{json}{
  morekeywords={},
  morecomment=[l]{},
  morestring=[b]"
}

\lstdefinelanguage{terra}{
  morekeywords={function, return, terra, end, var},
  morecomment=[l]{--},
  morestring=[b]"
}

\makeglossaries

\newglossaryentry{litval}{
  name=literal value,
  plural=literal values,
  description=A value that does not hold any pointer to dynamic memory.
}

\newglossaryentry{constexpr}{
  name=constexpr,
  description=A value or function that can be used in a constant expression.
}

\newglossaryentry{consteval}{
  name=constant evaluation,
  plural=constant evaluations,
  description=The evaluation of an expression that is performed at compile time.
}

\newacronym{api}{API}{Application Programming Interface}
\newacronym{ast}{AST}{Abstract Syntax Tree}
\newacronym{avx}{AVX}{Advanced Vector Extensions}
\newacronym{bf}{BF}{Brainfuck}
\newacronym{blas}{BLAS}{Basic Linear Algebra Subprograms}
\newacronym{ctpg}{CTPG}{Compile Time Parser Generator}
\newacronym{ctre}{CTRE}{Compile Time Regular Expression}
\newacronym{dsel}{DSEL}{Domain Specific Embedded Language}
\newacronym{dsl}{DSL}{Domain Specific Language}
\newacronym{erb}{ERB}{Embedded Ruby}
\newacronym{et}{ET}{Expression Template}
\newacronym{hpc}{HPC}{High Performance Computing}
\newacronym{jit}{JIT}{Just-In-Time}
\newacronym{nttp}{NTTP}{Non-Type Template Parameter}
\newacronym{pbg}{PBG}{pass-by-generator}
\newacronym{pcre}{PCRE}{Perl Compatible Regular Expression}
\newacronym{rpn}{RPN}{Reverse Polish Notation}
\newacronym{sfinae}{SFINAE}{Substitution Failure Is Not An Error}
\newacronym{tml}{TML}{Tiny Math Language}
\newacronym{tmp}{TMP}{Template Metaprogramming}

\newacronym{cpu}{CPU}{Central Processing Unit}
\newacronym{gpu}{GPU}{Graphical Processing Unit}

\newacronym{mimd}{SIMD}{Multiple Instruction Multiple Data stream}
\newacronym{misd}{SIMD}{Multiple Instruction Single Data stream}
\newacronym{simd}{SIMD}{Single Instruction Multiple Data stream}
\newacronym{sisd}{SIMD}{Single Instruction Single Data stream}

% ==============================================================================
\usepackage{subfiles} % Garder a la fin

\begin{document}

\begin{titlepage}

%\thispagestyle{empty}

\newgeometry{left=6cm,bottom=2cm, top=1cm, right=1cm}

\tikz[remember picture,overlay]
\node[opacity=1,inner sep=0pt] at (-13mm,-135mm){
  \includegraphics{ups/frame.pdf}
};

%*****************************************************
%******** NUMÉRO D'ORDRE DE LA THÈSE À COMPLÉTER *****
%******** POUR LE SECOND DÉPOT                   *****
%*****************************************************

\color{white}

\begin{picture}(0,0)
\put(-152,-743){\rotatebox{90}{\Large \textsc{THESE DE DOCTORAT}}} \\
\put(-120,-743){\rotatebox{90}{NNT : 2020UPASA001}}
\end{picture}

%*****************************************************
%******************** TITRE **************************
%*****************************************************

\flushright
\vspace{10mm} % à régler éventuellement
\color{Prune}

\fontsize{22}{26}\selectfont
  \Huge Techniques avanc\'ees de g\'en\'eration de code pour le parall\'elisme\\

\normalsize
\color{black}
\Large{\textit{Advanced techniques for parallel code generation}} \\
%*****************************************************


\fontsize{8}{12}\selectfont

\vspace{1.5cm}

\normalsize
\textbf{Thèse de doctorat de l'université Paris-Saclay} \\

\vspace{6mm}

% TODO
\small École doctorale n$^{\circ}$ d'accréditation, dénomination et sigle\\
\small Spécialité de doctorat: voir annexe\\
\small Graduate School : voir annexe. Référent : voir annexe \\
\vspace{6mm}

% TODO: Verifier l'unite de recherche
\footnotesize Thèse préparée dans la (ou les) unité(s) de recherche
\textbf{STIC} (voir annexe), sous la direction de \textbf{Joel FALCOU},
titre du directeur ou de la directrice de thèse \\
\vspace{15mm}

\textbf{Thèse soutenue à Paris-Saclay, le JJ mois AAAA, par}\\
\bigskip
\Large {\color{Prune} \textbf{Jules P\'ENUCHOT}} % Changer le Prénom et le NOM

%************************************
\vspace{\fill} % ALIGNER LE TABLEAU EN BAS DE PAGE
%************************************

\bigskip

\flushleft
\small {\color{Prune} \textbf{Composition du jury}}\\
{\color{Prune} \scriptsize {Membres du jury avec voix délibérative}} \\
\vspace{2mm}
\scriptsize
\begin{tabular}{|p{7cm}l}
\arrayrulecolor{Prune}
\textbf{Prénom NOM} &   Président ou Présidente\\
Titre, Affiliation & \\
\textbf{Prénom NOM} &  Rapporteur \& Examinateur / trice \\
Titre, Affiliation   &   \\
\textbf{Prénom NOM} &  Rapporteur \& Examinateur / trice \\
Titre, Affiliation  &   \\
\textbf{Prénom NOM} &  Examinateur ou Examinatrice \\
Titre, Affiliation   &   \\
\textbf{Prénom NOM} &  Examinateur ou Examinatrice \\
Titre, Affiliation   &   \\

\end{tabular}

\end{titlepage}

Remerciements

% TODO Merci les collegues et les parents
% TODO Merci les rapporteureuses et aux membres du jury
% TODO Merci les potes

% page des résumés à garder en 2ème page.
% Si les résumés sont trop longs pour tenir sur une seule et même page,
% on peut mettre un résumé par page
\thispagestyle{empty}
\newgeometry{top=1.5cm, bottom=1.25cm, left=2cm, right=2cm}

\noindent
\includegraphics[height=2.45cm]{ups/logo_STIC.png}
\vspace{1cm}
%*****************************************************

\small

\begin{mdframed}[linecolor=Prune,linewidth=1]

\textbf{Titre:} Techniques avanc\'ees de g\'en\'eration de code pour le
parall\'elisme

\noindent \textbf{Mots clés:} M\'etaprogrammation, compilation, C++ % TODO

\vspace{-.5cm}
\begin{multicols}{2}
\noindent \textbf{Résumé:} Mettre le sum ici % TODO
\end{multicols}

\end{mdframed}

\vspace{8mm}

\begin{mdframed}[linecolor=Prune,linewidth=1]

\textbf{Title:} Advanced techniques for parallel code generation

\noindent \textbf{Keywords:} Metaprogramming, compilation, C++ % TODO

\begin{multicols}{2}
\noindent \textbf{Abstract:} Mettre l'abstract ici % TODO
\end{multicols}
\end{mdframed}

\titleformat{\chapter}[hang]{\bfseries\Large\color{Prune}}{\thechapter\ -}{.1ex}
{\vspace{0.1ex}}[\vspace{1ex}]\titlespacing{\chapter}{0pc}{0ex}{0.5pc}

\titleformat{\section}[hang]{\bfseries\normalsize}{\thesection\ .}{0.5pt}
{\vspace{0.1ex}}[\vspace{0.1ex}]\titlespacing{\section}{1.5pc}{4ex plus .1ex minus .2ex}{.8pc}

\titleformat{\subsection}[hang]{\bfseries\small}{\thesubsection\ .}{1pt}
{\vspace{0.1ex}}[\vspace{0.1ex}]\titlespacing{\subsection}{3pc}{2ex plus .1ex minus .2ex}{.1pc}

\newgeometry{top=4cm, bottom=4cm, left=2cm, right=2cm}

\tableofcontents

\newgeometry{top=4cm, bottom=4cm, left=4cm, right=4cm}

% ------------------------------------------------------------------------------
% Commandes pour la these de Jules

% ------------------------------------------------------------------------------
% These de Jules

\part{Current state of metaprogramming for high performance computing}

\chapter*{
  Introduction
}

\subfile{introduction.tex}

\chapter{
  Metaprogramming
}

In this chapter, I will first give an overview of metaprogramming in
various languages. Then I will focus on the state of the art
of \cpp metaprogramming, and finally give examples of applications
of such techniques being used in the context of \gls{hpc} libraries.

\subfile{1-current-metaprogramming/1-metaprog-and-hpc-overview.tex}
\subfile{1-current-metaprogramming/2-cpp-constructs.tex}

\chapter{
  Code generation at low level
}

\subfile{1-current-metaprogramming/3-gemv.tex}

\part{C++ metaprogramming beyond templates}

\chapter{
  Compile time benchmarking methodology
}

With \gls{tmp} libraries like Eigen\cite{eigen}, Blaze\cite{blazelib},
or CTRE \cite{ctre} becoming more widespread,
there is an increasing interest for compile time computation.
These needs might grow
even larger as \cpp embeds more features over time to support and extend this
kind of practices, like compile time containers \cite{more-constexpr-containers}
or static reflection\cite{static-reflection}. However, there is still no clear
cut methodology to compare the performance impact of different metaprogramming
strategies.
As new \cpp features allow for new techniques with alleged
better compile time performance, no scientific process can back up those claims.

In this chapter, I introduce \textbf{ctbench}, which is a set of tools for
compile time benchmarking and analysis in \cpp. It provides developer-friendly
tools
to define and run benchmarks, then aggregate, filter out, and plot the data to
analyze them. As such, \ctbench aims to be a foundational layer of a proper
scientific methodology for analyzing compile time program behavior.
\\

ctbench puts an emphasis on software quality.
The goal was not just to develop a plotting tool for a single
compilation time analysis, but to provide a repeatable process
along with a robust implementation to improve compilation time
performance analysis as a whole.


% The use of a GitHub CI for building and testing the project guarantees that
% project remains functional at all times on all supported platforms.
% As of writing, the project is continuously built and tested for Ubuntu 23.04
% and Arch Linux. The test environment is fully reproducible as well thanks to
% the use of Docker for its setup.

\ctbench was presented at CPPP 2021 \cite{ctbench-cppp21},
and at Meeting \cpp 2022 \cite{meetingcpp22}. It was also published
at the Journal of Open Source Science \cite{Penuchot2023}.
It is currently available as an open source project at
\url{https://github.com/jpenuchot/ctbench}.
It is also available as a package through the Arch User Repository and vcpkg.

\subfile{2-compilation-time-analysis/1-state-of-the-art.tex}
\subfile{2-compilation-time-analysis/2-ctbench-design.tex}
\subfile{2-compilation-time-analysis/3-ctbench-in-action.tex}

% \subfile{2-compilation-time-analysis/features-and-design.tex}

\section{
  Conclusion
}

% Anbalyser le ct c'est important
% y'avait rien de satisfaisant surtout pour la science
% on en a un il est simple
% il permet des benchmarks reproductibles

Deep analysis of compile time scaling becomes necessary to understand
compile time performance of metaprogramming techniques.
Until now, no tool was really able to combine the depth of profiling data
analysis with variable size compile time benchmarking.
Moreover, there was no tool that could really fit in scientific work
where reproductibility is a necessity.

ctbench answers that need as a simple, powerful, and extensible
open-source solution that is peer-reviewed and distributed as a reusable
software package.

\chapter{
  Constexpr parsing for high performance computing
}
% \begin{abstract}
% Design of high performance, high abstraction libraries in \cpp often take
% advantages of techniques like template metaprogramming or \glspl{et}
% to design \glspl{dsel}. However, such techniques are limited by the natural syntax
% of \cpp. In this thesis, we explore the benefits of using \cpp23 compile time
% computation features to provide compile time string based \glspl{dsel} that can then
% use arbitrary syntax.
% \end{abstract}

\section{Introduction}

\cpp is often touted as a \textit{Zero-Cost Abstraction} langage due to some of
its design philosophy and its ability to compile abstraction to a very efficient
binary code. Some more radical techniques can be used to force the compiler to
interpret \cpp code as a \gls{dsel}. \Gls{tmp} is such a technique
and it spawned a large corpus of related idioms from compile time function
evaluation to lazy evaluation via \glspl{et}.

In the field of High Performance Computing, \cpp users are often driven to use
libraries built on top of those idioms like Eigen\cite{eigen} or
Blaze\cite{blazelib,iglberger2012_2}. They all suffer from a major limitation:
by being tied to the natural \cpp syntax, they can't express nor embed arbitrary
languages.

In this thesis, we try to demonstrate that the new features of \cpp23 related to
compile time programming are able to help developers designing \glspl{dsel} with
arbitrary syntax by leveraging \gls{constexpr} computations, compile time dynamic
objects and lazy evaluation through lambda functions. After contextualizing our
contribution in the general \glspl{dsel} domain, this thesis will explain the core
new techniques enabled by \cpp23 and how we can apply to build two different
\glspl{dsel} with their own non-\cpp syntax. We'll also explore the performances of
said \glspl{dsel} in term of compile time to assess their usability in realistic code.

\subfile{3-new-approaches-to-metaprogramming/1-technical-background.tex}
\subfile{3-new-approaches-to-metaprogramming/2-constexpr-codegen-techniques.tex}

\chapter{
  Applied constexpr parsing
}

Now that the idea of using \gls{constexpr} functions to generate code
was proven to work, it is time to put it into practice and observe
the viability of its integration in a high performance computing
development cycle.

I will present two examples in this chapter:
the first one is a \gls{constexpr} Brainfuck parser that serves as a simple
use case for experimenting different methods to generate code from
\gls{constexpr} \glspl{ast}.

These observations guided the implementation of the second example, which is a
\gls{constexpr} parser for simple math languages. If features a code generation
backend that supports high performance code generation via Blaze.

\subfile{3-new-approaches-to-metaprogramming/3-brainfuck.tex}
\subfile{3-new-approaches-to-metaprogramming/4-math-parsing.tex}

\section{Conclusion}

The Brainfuck example shows that using \gls{constexpr} programming to embed
large programs written in foreign languages in \cpp can be achieved.
\gls{ast} serialization makes it possible to store parsing results
and avoid running into quadratic compilation time issues, as long as
the code generation methods being .

The case of \gls{tml} also shows that \gls{ast} serialization
by itself is not a magic bullet to avoid running into compilation time
scaling issues.
The compilation time offset of \gls{tml} parsing, despite being reasonable
within the bounds of expected use cases for \gls{tml}, was quadratic.

These results show that more effort is needed to determine which metaprogramming
techniques should be avoided to avoid runaway compilation times,
and how \cpp library code or modifications to the \cpp language itself
could facilitate scalable code generation from \gls{constexpr} function results.

\chapter*{Conclusion}

\subfile{conclusion.tex}

\clearpage
\printglossaries

\clearpage
\printbibliography

\chapter{Appendix}

\appendix

\subfile{3-new-approaches-to-metaprogramming/appendix.tex}

\end{document}
