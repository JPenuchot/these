\documentclass[../main]{subfiles}
\begin{document}

\section{
  Constexpr abstract syntax trees and code generation
}

The addition of \gls{constexpr} memory and virtual function calls allows
\gls{constexpr} implementation of \glspl{ast}, akin to those found
in modern \cpp compiler frontends like Clang.
However, these \glspl{ast} cannot be used directly as \glspl{nttp}
for generating structures and functions.

This section will cover the implementation of \gls{constexpr} \glspl{ast},
and how to work around the limitations that prevent their direct use as
\glspl{nttp}, either through functional wrapping techniques, or through
their convertion of into values that satisfy \gls{nttp} requirements.

\subsection{
  Code generation from pointer tree data structures
}

\label{lbl:ptr-tree-codegen}

In this subsection, we introduce three techniques that will allow us to use
a pointer tree generated from a \gls{constexpr} function as a template parameter
for code generation.

To illustrate them, we use a minimalistic use case.
A generator function returns a pointer tree representation of addition and
constant nodes, which we will use to generate functions that evaluate
the tree itself.

\begin{lstlisting}[
  language=C++,
  label=lst:tree_generation,
  caption=\lstinline{tree_t} type definition
]{}
/// Enum type to differentiate nodes
enum node_kind_t {
  constant_v,
  add_v,
};

/// Node base type
struct node_base_t {
  node_kind_t kind;

  constexpr node_base_t(node_kind_t kind_) : kind(kind_) {}
  constexpr virtual ~node_base_t() = default;
};

/// Tree pointer type
using tree_ptr_t = std::unique_ptr<node_base_t>;

/// Checks that an object is a tree generator
template <typename T>
concept tree_generator = requires(T fun) {
  { fun() } -> std::same_as<tree_ptr_t>;
};

/// Constant node type
struct constant_t : node_base_t {
  int value;
  constexpr constant_t(int value_)
      : node_base_t(constant_v), value(value_) {}
};

/// Addition node type
struct add_t : node_base_t {
  tree_ptr_t left;
  tree_ptr_t right;
  constexpr add_t(tree_ptr_t left_, tree_ptr_t right_)
      : node_base_t(add_v), left(std::move(left_)),
        right(std::move(right_)) {}
};

/// Generates an arbitrary tree
constexpr tree_ptr_t gen_tree() {
  return std::make_unique<add_t>(
      std::make_unique<add_t>(std::make_unique<constant_t>(1),
                              std::make_unique<constant_t>(2)),
      std::make_unique<constant_t>(3));
}
\end{lstlisting}

Listing \ref{lst:tree_generation} shows the type definitions, a concept to
match tree generator functions, as well as the generator function itself.
This is a common way to represent trees in \cpp, but the limits mentioned in
\ref{lbl:constexpr-programming} make it impossible to use the result of
\lstinline{gen_tree} directly as a template parameter to generate code.
\\

The techniques we will use to pass the result as a template parameter are:

\begin{itemize}
\item passing functions that return nodes as \glspl{nttp},
\item converting the tree into an expression template representation,
\item serializing the tree into a dynamically allocated array,
      then converting the dynamic array into a static array that can be used as
      a \gls{nttp},
\end{itemize}

The compilation performance measurements in \ref{lbl:compile-time-eval} will
rely on the same data passing techniques, but with more complex examples such
as embedded compilation of Brainfuck programs, and of \LaTeX math formulae
into high performance math computation kernels.

\subsubsection{
  Pass-by-generator
}

\label{lbl:pbg-technique}

One way to use dynamically allocated data structures as template parameters
is to pass their generator functions instead of their values.
You may have noted in listing \ref{lst:cx-examples} that depsite
\lstinline{generate}'s return value being non-literal, the function itself
can be passed as a \gls{nttp}.

Its result can be used to produce literal \gls{constexpr} results,
and the function itself can be used in generator lambda functions
defined at compile-time.

\begin{lstlisting}[
  language=C++,
  label=lst:pass-by-generator-example,
  caption=Pass-by-generator
]{}
/// Accumulates the value from a tree returned by TreeGenerator.
/// TreeGenerator() is expected to return a std::unique_ptr<tree_t>.
template <tree_generator auto Fun> constexpr auto codegen() {
  static_assert(Fun() != nullptr, "Ill-formed tree");

  constexpr node_kind_t Kind = Fun()->kind;

  if constexpr (Kind == add_v) {
    // Recursive codegen for left and right children:
    // for each child, a generator function is generated
    // and passed to codegen.
    auto eval_left = codegen<[]() {
      return static_cast<add_t &&>(*Fun()).left;
    }>();
    auto eval_right = codegen<[]() {
      return static_cast<add_t &&>(*Fun()).right;
    }>();

    return [=]() { return eval_left() + eval_right(); };
  }

  else if constexpr (Kind == constant_v) {
    constexpr auto Constant =
        static_cast<constant_t const &>(*Fun()).value;
    return [=]() { return Constant; };
  }
}
\end{lstlisting}

In listing \ref{lst:pass-by-generator-example}, we show how a \gls{constexpr}
pointer tree result can be visited recusively using generator lambdas
to pass the subnodes' values, and used to generate code.

This technique is fairly simple to implement as it does not require any
transformation into an ad hoc data structure to pass the tree as a type
or \gls{nttp}.

The downside of using this value passing technique is that the
number of calls of the generator function is proportional to the
number of nodes. Experiments in \ref{lbl:compile-time-eval} highlight
the scaling issues induced by this code generation method.
And while it is very quick to implement, there are still difficulties
related to \gls{constexpr} memory constraints and compiler or library support.

Note that GCC 13.2.1 is still unable to compile such code.
And while Clang 17.0.6 supports it, getting a this technique to work can be
difficult as \gls{constexpr} memory rules are enforced in an erratic way,
and often with poorly explained compiler errors.

For example, if the expression in an \lstinline{if constexpr}
condition involves \gls{constexpr} allocated memory, the result of the condition
must be stored in an intermediate \gls{constexpr} variable to avoid a compiler
error:

\begin{lstlisting}[
  language=c++
]{}
constexpr std::vector<int> foo() {
  return {{1, 2, 3, 4, 5}};
}

// Compilation error
if constexpr (foo().size() == 5);

// OK
if constexpr (constexpr bool val =
                  foo().size() == 5;
              val);
\end{lstlisting}

With no \gls{constexpr} intermediate variable, Clang tells that the condition
expression is not a constant expression, despite the condition expression being
usable to initialize a \gls{constexpr} variable.

\subsubsection{
  Pass-by-generator + ET
}

\label{lbl:pbg-et-technique}

Why through? Interoperability with type-based metaprogramming libraries.

Types:

\begin{lstlisting}[
  language=C++
]{}
/// Type representation of a constant
template <int Value> struct et_constant_t {};
/// Type representation of an addition
template <typename Left, typename Right> struct et_add_t {};
\end{lstlisting}

Codegen:

\begin{lstlisting}[
  language=C++
]{}
/// Accumulates the value from a tree returned by TreeGenerator.
/// TreeGenerator() is expected to return a std::unique_ptr<tree_t>.
template <tree_generator auto Fun>
constexpr auto to_expression_template() {
  static_assert(Fun() != nullptr, "Ill-formed tree");

  constexpr node_kind_t Kind = Fun()->kind;

  if constexpr (Kind == add_v) {
    // Recursive type generation using the pass-by-generator technique
    using TypeLeft = decltype(to_expression_template<[]() {
      return static_cast<add_t &&>(*Fun()).left;
    }>());
    using TypeRight = decltype(to_expression_template<[]() {
      return static_cast<add_t &&>(*Fun()).right;
    }>());

    return et_add_t<TypeLeft, TypeRight>{};
  }

  else if constexpr (Kind == constant_v) {
    constexpr auto Value = static_cast<constant_t &>(*Fun()).value;
    return et_constant_t<Value>{};
  }
}

template <int Value>
constexpr auto codegen_impl(et_constant_t<Value> /*unused*/) {
  return []() { return Value; };
}

template <typename ExpressionLeft, typename ExpressionRight>
constexpr auto
codegen_impl(et_add_t<ExpressionLeft, ExpressionRight> /*unused*/) {
  auto eval_left = codegen_impl(ExpressionLeft{});
  auto eval_right = codegen_impl(ExpressionRight{});
  return [=]() { return eval_left() + eval_right(); };
}

template <tree_generator auto Fun> constexpr auto codegen() {
  using ExpressionTemplate = decltype(to_expression_template<Fun>());
  return codegen_impl(ExpressionTemplate{});
}
\end{lstlisting}

\subsubsection{
  FLAT
}

\label{lbl:flat-technique}

To overcome the performance issues of the previously introduced techniques,
we can try a different approach. Instead of passing trees as they are,
they can be transformed into static arrays which can be used as \glspl{nttp}.

\begin{lstlisting}[language=C++, label=lst:flat_struct_def]{}
/// Serialized representation of a constant
struct flat_constant_t {
  int value;
};

/// Serialized representation of an addition
struct flat_add_t {
  std::size_t left;
  std::size_t right;
};

/// Serialized representation of a node
using flat_node_t = std::variant<flat_add_t, flat_constant_t>;

/// Defining max std::size_t value as an equivalent to std::nullptr.
constexpr std::size_t null_index = std::size_t(0) - 1;
\end{lstlisting}

This requires the tree to be serialized first.
For the sake of demonstration, we will serialize the tree into an ad hoc
data representation that is identical to the original one, except pointers
are replaced with \lstinline{std::size_t} indexes, and \lstinline{nullptr}
is replaced with an arbitrary value called \lstinline{null_index} as shown in
listing \ref{lst:flat_struct_def}.

Heritage polymorphism is also replaced by \lstinline{std::variant}

These nodes will be stored in \lstinline{std::vector} containers,
and the indexes will refer to the position of other nodes within the container.
Note that our tree nodes are not polymorphic. If needed,
\lstinline{std::variant} could have been used to have polymorphic nodes in
the serialized representation.

\begin{lstlisting}[
  language=C++,
  label=lst:flat_backend_serialization,
  caption=\gls{constexpr} tree serialization implementation
]{}
/// Serializes the current subtree and returns
/// the top node's index to the caller.
constexpr std::size_t serialize_impl(tree_ptr_t const &top,
                                     std::vector<flat_node_t> &out) {
  // nullptr translates directly to null_index
  if (top == nullptr) {
    return null_index;
  }

  // Allocating space for the destination node
  std::size_t dst_index = out.size();
  out.emplace_back();

  if (top->kind == add_v) {
    auto const &typed_top = static_cast<add_t const &>(*top);

    // Serializing left and right subtrees,
    // initializing the new node
    out[dst_index] = {
        flat_add_t{.left = serialize_impl(typed_top.left, out),
                   .right = serialize_impl(typed_top.right, out)}};
  }

  if (top->kind == constant_v) {
    auto const &typed_top = static_cast<constant_t const &>(*top);
    out[dst_index] = {flat_constant_t{.value = typed_top.value}};
  }

  return dst_index;
}

/// Returns a serialized representation of a pointer tree.
constexpr std::vector<flat_node_t> serialize(tree_ptr_t const &tree) {
  std::vector<flat_node_t> result;
  serialize_impl(tree, result);
  return result;
}
\end{lstlisting}

Listing \ref{lst:flat_backend_serialization} shows the implementation of the
serialization step. The implementation itself has nothing particular, except for
the functions being \gls{constexpr}.

Once the data is serialized, it can be converted into a static array container.
This can be done because the generator function is \gls{constexpr}, therefore it
can be used to produce \gls{constexpr} values. The size of the resulting
\lstinline{std::vector} can be stored at compile time to set the size
of a static array.

\begin{lstlisting}[
  language=C++,
  label=lst:serialize_as_array,
  caption=Definition of \lstinline{serialize_as_array}
]{}
/// Evaluates a tree generator function into a serialized array.
template <tree_generator auto Fun>
constexpr auto serialize_as_array() {
  constexpr std::size_t Size = serialize(Fun()).size();
  std::array<flat_node_t, Size> result;
  std::ranges::copy(serialize(Fun()), result.begin());
  return result;
}
\end{lstlisting}

Listing \ref{lst:serialize_as_array} shows the implementation of a helper
function that takes a \gls{constexpr} tree generator function as an input,
serializes the result, and returns it as a static array.

Static arrays are not dynamically allocated, therefore they can be used as
\glspl{nttp} if their values do not hold pointers to
\gls{constexpr} allocated memory either.
In our case, the elements only hold integers, so a \lstinline{std::array} of
\lstinline{flat_node_t} elements can be used as a \glspl{nttp}.

\begin{lstlisting}[
  language=C++,
  label=lst:flat_codegen,
  caption=Code generation implementation for the flat backend
]{}
/// Generates code from a tree serialized into a static array,
/// with CurrentIndex being the index of the starting node which
/// defaults to 0, ie. the top node of the whole tree.
template <auto const &Tree, std::size_t Index>
constexpr auto codegen_aux() {
  static_assert(Index != null_index, "Ill-formed tree (null index)");

  if constexpr (std::holds_alternative<flat_add_t>(Tree[Index])) {
    constexpr auto top = std::get<flat_add_t>(Tree[Index]);

    // Recursive code generation for left and right children
    auto eval_left = codegen_aux<Tree, top.left>();
    auto eval_right = codegen_aux<Tree, top.right>();

    // Code generation for current node
    return [=]() { return eval_left() + eval_right(); };
  }

  else if constexpr (std::holds_alternative<flat_constant_t>(
                         Tree[Index])) {
    constexpr auto top = std::get<flat_constant_t>(Tree[Index]);
    constexpr int Value = top.value;
    return []() { return Value; };
  }
}

/// Stores serialized representations of tree generators' results.
template <tree_generator auto Fun>
static constexpr auto tree_as_array = serialize_as_array<Fun>();

/// Takes a tree generator function as non-type template parameter
/// and generates the lambda associated to it.
template <tree_generator auto Fun> constexpr auto codegen() {
  return codegen_aux<tree_as_array<Fun>, 0>();
}
\end{lstlisting}

To complete the implementation, we must implement a code generation function
that accepts a serialized tree as an input as shown in listing
\ref{lst:flat_codegen}. Note that this function is almost identical to
the one shown in listing \ref{lst:pass-by-generator-example}.
The major difference is that \lstinline{TreeGenerator} is called only
twice regardless of the size of the tree. This allows much better scaling
as we will see in \ref{lbl:compile-time-eval}.
The downside is that it requires the implementation of an ad hoc data
structure and a serialization function, which might be more or less complex
depending on the complexity of the original tree structure.

\subsection{
  Using algorithms with serialized outputs
}
\label{lbl:codegen-from-rpn}

Parsing algorithms may output serialized data. In this case, the serialization
step described in \ref{lbl:flat-technique} is not needed, and the result
can be converted into a static array.
This makes the code generation process rather straightforward as no complicated
transformation is needed, while still scaling decently as we will see in
\ref{lbl:compile-time-eval} where we will be using a
Shunting Yard parser \cite{shunting-yard} to parse math formulae to a
\gls{rpn}, which is its postfix notation.

Once converted into its postfix notation,
a formula can be read using the following method:

\begin{itemize}
\item read symbols in order,
\item put constants and variables on the top of a stack,
\item when a function $f$ or operator of arity $N$ is being read,
      unstack $N$ values and stack the result of $f$ applied
      to the $N$ operands.
\end{itemize}

\begin{figure}
\begin{tabular}{|l|l|l|}
\hline
Current token & Action & Stack \\
\hline
\lstinline|2| & Stack \lstinline|2| & \lstinline|2| \\
\lstinline|3| & Stack \lstinline|3| & \lstinline|2|, \lstinline|3| \\
\lstinline|2| & Stack \lstinline|2| & \lstinline|2|, \lstinline|3|, \lstinline|2| \\
\lstinline|*| & Multiply \lstinline|2| and \lstinline|3| & \lstinline|2|, \lstinline|6| \\
\lstinline|+| & Add \lstinline|2| and \lstinline|6| & \lstinline|8| \\
\hline
\end{tabular}
\caption{\gls{rpn} formula reading example}
\label{fig:rpn-reading-example}
\end{figure}

Figure \ref{fig:rpn-reading-example} shows a formula reading example with
the formula \lstinline{2 + 3 * 2}, or \lstinline{2 3 2 * +} in
reverse polish notation.

Starting from there, we will see how code can be generated using
RPN representations of addition trees in \cpp.

\begin{lstlisting}[
  language=C++,
  label=lst:rpn_base_definitions,
  caption=RPN example base type and function definitions
]{}
/// Type for RPN representation of a constant
struct rpn_constant_t {
  int value;
};

/// Type for RPN representation of an addition
struct rpn_add_t {};

/// Type for RPN representation of an arbitrary symbol
using rpn_node_t = std::variant<rpn_constant_t, rpn_add_t>;

/// RPN equivalent of gen_tree
constexpr std::vector<rpn_node_t> gen_rpn_tree() {
  return {rpn_constant_t{1}, rpn_constant_t{2}, rpn_add_t{},
          rpn_constant_t{3}, rpn_add_t{}};
}
\end{lstlisting}

In listing \ref{lst:rpn_base_definitions}, we have the type definitions for an
RPN representation of an addition tree as well as \lstinline{gen_rpn_tree} which
returns an RPN equivalent of \lstinline{gen_tree}'s result.

Similar to the flat backend, an \lstinline{eval_as_array} takes care
of evaluating the \lstinline{std::vector} result into a statically
allocated array.

\begin{lstlisting}[
  language=C++,
  caption=Codegen implementation for RPN formulae,
  label=lst:rpn_codegen
]{}
/// Codegen implementation.
/// Reads tokens one by one, updates the stack consequently
/// by consuming and/or stacking operands.
/// Operands are functions that evaluate parts of the subtree.
template <auto const &RPNTree, std::size_t Index = 0>
constexpr auto codegen_impl(kumi::product_type auto stack) {
  // The end result is the last top stack operand
  if constexpr (Index == RPNTree.size()) {
    static_assert(stack.size() == 1, "Invalid tree");
    return kumi::back(stack);
  }

  else if constexpr (std::holds_alternative<rpn_constant_t>(
                         RPNTree[Index])) {
    // Append the constant operand
    auto new_operand = [=]() {
      return get<rpn_constant_t>(RPNTree[Index]).value;
    };
    return codegen_impl<RPNTree, Index + 1>(
        kumi::push_back(stack, new_operand));
  }

  else if constexpr (std::holds_alternative<rpn_add_t>(
                         RPNTree[Index])) {
    // Fetching 2 top elements and popping them off the stack
    auto left = kumi::get<stack.size() - 1>(stack);
    auto right = kumi::get<stack.size() - 2>(stack);
    auto stack_remainder = kumi::pop_back(kumi::pop_back(stack));

    // Append new operand and process next element
    auto new_operand = [=]() { return left() + right(); };
    return codegen_impl<RPNTree, Index + 1>(
        kumi::push_back(stack_remainder, new_operand));
  }
}

/// Stores static array representations of RPN generators' results.
template <auto Fun>
static constexpr auto rpn_as_array = eval_as_array<Fun>();

/// Code generation function.
template <auto Fun> constexpr auto codegen() {
  return codegen_impl<rpn_as_array<Fun>, 0>(kumi::tuple{});
}
\end{lstlisting}

In listing \ref{lst:rpn_codegen}, we have function definitions for the
implementation of \lstinline{codegen} which takes an RPN generator function,
and generates a function that evaluates the tree.

The code generation process happens by updating an operand stack represented
by a \lstinline{kumi::tuple}. It is a standard-like tuple type with additional
element access, extraction, and modification functions.
We are using it to store functions that evaluate parts of the subtree.

Symbols are read in order, and the stack is updated depending on the symbol:

\begin{itemize}
\item When a terminal is read, a function that evaluates its value is stacked.
In this case terminals are simply constants, but they can be anything a \cpp
lambda can return such as a variable or another function's result.

\item When a function or operator of arity $N$ is read, $N$ operands will be
consumed from the top of the stack and a new operand will be stacked.
The new operand evaluates the consumed operands passed to the function
corresponding to the symbol being read.
\end{itemize}

Once all symbols are read, there should be only one operand remaining
on the stack: the function that evaluates the whole tree.

\section{
  Conclusion
}

% TODO: Reflow this conclusion

The presented code snippets already allow us to draw a few observations.

\begin{itemize}

\item
Code generation from \gls{constexpr} dynamic structures is not a trivial process.
It still requires advanced \cpp knowledge to understand the limits of what
is or isn't possible to do with \glspl{nttp} and \gls{constexpr} functions.

The code generation examples themselves do not show the difficulties related to
\cpp \gls{constexpr} restrictions, and the inconsistent enforcement
of those restrictions by Clang.

\item
The easiest way to get around \gls{nttp} constraints, \ie the \gls{pbg}
technique, is very costly in terms of compilation times.

Section \ref{lbl:compile-time-eval} covers that scaling issue in more detail.
The overall assesment is that this technique can be used for small metaprograms,
but it fails to scale properly as larger ones are being considered
due to its quadratic compilation time complexity.

In our Brainfuck metacompilation examples, we were able to trigger Clang's
timeout using this technique when compiling a Mandelbrot visualizer
whereas the so-called "flat" backend was able to generate the program in less
than a minute.

\item
Compiler support for \gls{constexpr} programming and constant evaluation in general
is still very inconsistent across compilers. GCC still has issues with
\lstinline{if constexpr} where templates are instantiated even when they are
in a discarded statement.

\item
Library support is also limited.
Most of the containers from the \cpp standard library are not usable in
\gls{constexpr} functions simply because the \cpp standard lacks \gls{constexpr}
qualfications for their methods (and \gls{constexpr} qualification is not implicit).

As of today, the only exceptions are the main containers such as
\lstinline{std::vector}, \lstinline{std::string} (since \cpp20),
or \lstinline{std::unique_ptr} (since \cpp23).

The C'est \cite{cest} library however provides more standard-like containers
that are usable in \gls{constexpr} functions in \cpp20, which is enough to make
the previous examples run on Clang in \cpp20.

\end{itemize}

So far we can conclude that while being doable, code generation from \gls{constexpr}
function results with dynamic memory is still not accessible to all \cpp
programmers.

An effort on the language itself would be needed to allow easier data passing
from \gls{constexpr} allocated memory to \gls{nttp}.

Alternatively, the language could allow code generation

\end{document}
