\documentclass[../main]{subfiles}
\begin{document}

\clearpage%

\begin{lstlisting}[
  language=C++,
  label=lst:tree_generation,
  caption=\lstinline{tree_t} type definition
]{}
/// Enum type to differentiate nodes
enum node_kind_t {
  constant_v,
  add_v,
};

/// Node base type
struct node_base_t {
  node_kind_t kind;

  constexpr node_base_t(node_kind_t kind_) : kind(kind_) {}
  constexpr virtual ~node_base_t() = default;
};

/// Tree pointer type
using tree_ptr_t = std::unique_ptr<node_base_t>;

/// Checks that an object is a tree generator
template <typename T>
concept tree_generator = requires(T fun) {
  { fun() } -> std::same_as<tree_ptr_t>;
};

/// Constant node type
struct constant_t : node_base_t {
  int value;
  constexpr constant_t(int value_)
      : node_base_t(constant_v), value(value_) {}
};

/// Addition node type
struct add_t : node_base_t {
  tree_ptr_t left;
  tree_ptr_t right;
  constexpr add_t(tree_ptr_t left_, tree_ptr_t right_)
      : node_base_t(add_v), left(std::move(left_)),
        right(std::move(right_)) {}
};

/// Generates an arbitrary tree
constexpr tree_ptr_t gen_tree() {
  return std::make_unique<add_t>(
      std::make_unique<add_t>(std::make_unique<constant_t>(1),
                              std::make_unique<constant_t>(2)),
      std::make_unique<constant_t>(3));
}
\end{lstlisting}

\clearpage%

\section{
  Constexpr abstract syntax trees and code generation
}
\label{lbl:ast-codegen}

The addition of \gls{constexpr} memory and virtual function calls allows
\gls{constexpr} implementation of \glspl{ast}, akin to those found
in modern \cpp compiler frontends like Clang.
However, these \glspl{ast} cannot be used directly as \glspl{nttp}
for generating structures and functions.

This section will cover the implementation of \gls{constexpr} \glspl{ast},
and techniques to work around the limitations that prevent their direct use as
\glspl{nttp} either through functional wrapping techniques, or through
their convertion of into values that satisfy \gls{nttp} requirements.

\subsection{
  Code generation from pointer tree data structures
}

\label{lbl:ptr-tree-codegen}

In this subsection, we introduce three techniques that will allow us to use
a pointer tree generated from a \gls{constexpr} function as a template parameter
for code generation.

To illustrate them, we use a minimalistic use case.
A generator function returns a pointer tree representation of addition and
constant nodes, which we will use to generate functions that evaluate
the tree itself.

Listing \ref{lst:tree_generation} shows the type definitions, a concept to
match tree generator functions, as well as the generator function itself.
This is a common way to represent trees in \cpp, but the limits mentioned in
\ref{lbl:constexpr-programming} make it impossible to use the result of
\lstinline{gen_tree} directly as a template parameter to generate code.
\\

The techniques we will use to pass the result as a template parameter are:

\begin{itemize}
\item passing functions that return nodes as \glspl{nttp},
\item converting the tree into an expression template representation,
\item serializing the tree into a dynamically allocated array,
      then converting the dynamic array into a static array that can be used as
      a \gls{nttp},
\end{itemize}

The compilation performance measurements in \ref{lbl:compile-time-eval} will
rely on the same data passing techniques, but with more complex examples such
as embedded compilation of Brainfuck programs, and of \LaTeX math formulae
into high performance math computation kernels.

\subsubsection{
  Pass By Generator (PBG)
}

\label{lbl:pbg-technique}

One way to use dynamically allocated data structures as template parameters
is to pass their generator functions instead of their values.
You may have noted in listing \ref{lst:cx-examples} that despite
\lstinline{generate}'s return value being non-literal, the function itself
can be passed as a \gls{nttp}.

Its result can be used to produce literal \gls{constexpr} results,
and the function itself can be used in generator lambda functions
defined at compile-time.

\begin{lstlisting}[
  language=C++,
  label=lst:pass-by-generator-example,
  caption=Code generation with the pass by generator technique
]{}
/// Accumulates the value from a tree returned by
/// TreeGenerator. TreeGenerator() is expected to return a
/// std::unique_ptr<tree_t>.
template <tree_generator auto Fun>
constexpr auto codegen() {
  static_assert(Fun() != nullptr, "Ill-formed tree");

  constexpr node_kind_t Kind = Fun()->kind;

  if constexpr (Kind == add_v) {
    // Recursive codegen for left and right children:
    // for each child, a generator function is generated
    // and passed to codegen.
    auto eval_left = codegen<[]() {
      return static_cast<add_t &&>(*Fun()).left;
    }>();
    auto eval_right = codegen<[]() {
      return static_cast<add_t &&>(*Fun()).right;
    }>();

    return [=]() { return eval_left() + eval_right(); };
  }

  else if constexpr (Kind == constant_v) {
    constexpr auto Constant =
        static_cast<constant_t const &>(*Fun()).value;
    return [=]() { return Constant; };
  }
}
\end{lstlisting}

In listing \ref{lst:pass-by-generator-example}, we show how a \gls{constexpr}
pointer tree result can be visited recusively using generator lambdas
to pass the subnodes' values, and used to generate code.

This technique is fairly simple to implement as it does not require any
transformation into an ad hoc data structure to pass the tree as a type
or \gls{nttp}.

The downside of using this value passing technique is that the
number of calls of the generator function is proportional to the
number of nodes. Experiments in \ref{lbl:compile-time-eval} highlight
the scaling issues induced by this code generation method.
And while it is very quick to implement, there are still difficulties
related to \gls{constexpr} memory constraints and compiler or library support.

Note that GCC 13.2.1 is still unable to compile such code.
And while Clang 17.0.6 supports it, getting this technique to compile can be
difficult as \gls{constexpr} memory rules are enforced in an erratic way,
and often with poorly explained compiler errors.

\clearpage%

For example, if the expression in an \lstinline{if constexpr}
condition involves \gls{constexpr} allocated memory, the result of the condition
must be stored in an intermediate \gls{constexpr} variable to avoid a compiler
error:

\begin{lstlisting}[
  language=c++
]{}
constexpr std::vector<int> foo() {
  return {{1, 2, 3, 4, 5}};
}

// Compilation error
if constexpr (foo().size() == 5);

// OK
if constexpr (constexpr bool val =
                  foo().size() == 5;
              val);
\end{lstlisting}

With no \gls{constexpr} intermediate variable, Clang tells that the condition
expression is not a constant expression, despite the condition expression being
usable to initialize a \gls{constexpr} variable.

\subsubsection{
  Pass by generator and expression templates
}

\label{lbl:pbg-et-technique}

Another method consists in transforming \gls{constexpr} \glspl{ast}
into \glspl{et} as an \gls{ir}. This serves as a proof that making
\gls{constexpr} programming interoperable with older metaprogramming
paradigms is feasible, but it will also help us assess the viability
of using the \gls{pbg} method in coordination with \glspl{et}.

The first step consists in creating the \gls{ir} itself, which is a type-based
representation of the \gls{ast}.

\begin{lstlisting}[
  language=C++,
  caption=cap,
  label=lst:pbg-et-ir
]{}
/// Type representation of a constant
template <int Value> struct et_constant_t {};
/// Type representation of an addition
template <typename Left, typename Right> struct et_add_t {};
\end{lstlisting}


Listing \ref{lst:pbg-et-ir} shows the type-based \gls{ir} of the \gls{ast}
representation from listing \ref{lst:tree_generation}. It is meant to be
a one to one reimplementation of its value-based equivalent.

From there, we can implement a metaprogram that transforms a \gls{constexpr}
\gls{ast} into this new type-based representation.

\clearpage%

\begin{lstlisting}[
  language=C++,
  caption=AST to Expression Template transformation,
  label=lst:to_expression_template-impl
]{}
template <tree_generator auto Fun>
constexpr auto to_expression_template() {
  static_assert(Fun() != nullptr, "Ill-formed tree");

  constexpr node_kind_t Kind = Fun()->kind;

  if constexpr (Kind == add_v) {
    using TypeLeft =
        decltype(to_expression_template<[]() {
          return static_cast<add_t &&>(*Fun()).left;
        }>());
    using TypeRight =
        decltype(to_expression_template<[]() {
          return static_cast<add_t &&>(*Fun()).right;
        }>());

    return et_add_t<TypeLeft, TypeRight>{};
  }

  else if constexpr (Kind == constant_v) {
    constexpr auto Value =
        static_cast<constant_t &>(*Fun()).value;
    return et_constant_t<Value>{};
  }
}

template <int Value>
constexpr auto
codegen_impl(et_constant_t<Value> /*unused*/) {
  return []() { return Value; };
}

template <typename ExpressionLeft,
          typename ExpressionRight>
constexpr auto
codegen_impl(et_add_t<ExpressionLeft,
                      ExpressionRight> /*unused*/) {
  auto eval_left = codegen_impl(ExpressionLeft{});
  auto eval_right = codegen_impl(ExpressionRight{});
  return [=]() { return eval_left() + eval_right(); };
}

template <tree_generator auto Fun>
constexpr auto codegen() {
  using ExpressionTemplate =
      decltype(to_expression_template<Fun>());
  return codegen_impl(ExpressionTemplate{});
\end{lstlisting}

\clearpage%

As shown in listing \ref{lst:to_expression_template-impl}, it is very similar
to the transformation code in \ref{lbl:pbg-technique} since it relies on
the same technique to traverse the \gls{ast}.
However, its output is not a \cpp lambda but an \gls{et}, and as such
proves that \gls{constexpr} programming is interoperable with existing
type-based paradigms, even when \gls{constexpr} allocated memory is involved.

It is worth mentioning that both this technique and the previous one induce
very high compilation times, as we will see in \ref{lbl:bf-parsing-and-codegen}.

\subsubsection{
  FLAT - AST serialization
}

\label{lbl:flat-technique}

Alternatively, we can use value-based metaprogramming to generate code.
Instead of passing trees as they are,
they can be transformed into static arrays which can be used as \glspl{nttp}.

This transformation might sound odd at first: the generated \glspl{ast} are
dynamic structures that need dynamic memory allocation, and static arrays
have a fixed size. However, because the size of the output is known
at compile time, it is possible to determine the size of the serialized
\gls{ast} in a first step to determine the size of the array,
and then transfer the serialized content into the array.
\\

In short: "static" array sizes are static at run-time but not at compile-time.
This can be exploited to generate values that meet the requirements
of \glspl{nttp} but still contain the same data as the \glspl{ast} they were
generated from.

\begin{lstlisting}[language=C++, label=lst:flat_struct_def]{}
/// Serialized representation of a constant
struct flat_constant_t {
  int value;
};

/// Serialized representation of an addition
struct flat_add_t {
  std::size_t left;
  std::size_t right;
};

/// Serialized representation of a node
using flat_node_t =
    std::variant<flat_add_t, flat_constant_t>;

/// null_index replaces std::nullptr.
constexpr std::size_t null_index = std::size_t(0) - 1;
\end{lstlisting}

The first step consists in implementing the \gls{ir} nodes which are similar
to the \gls{ast} nodes in which pointers are replaced with
\lstinline{std::size_t} indexes,
and \lstinline{nullptr} is replaced with an arbitrary value called
\lstinline{null_index}, as shown in listing \ref{lst:flat_struct_def}.
This \gls{ir} also replaces heritage polymorphism with the use of
\lstinline{std::variant}.
These nodes are then stored in \lstinline{std::vector} containers,
and the indexes refer to the positions of other nodes within the same container.

\begin{lstlisting}[
  language=C++,
  label=lst:flat_backend_serialization,
  caption=constexpr AST serialization
]{}
constexpr std::size_t
serialize_impl(tree_ptr_t const &top,
               std::vector<flat_node_t> &out) {
  // nullptr translates directly to null_index
  if (top == nullptr) {
    return null_index;
  }

  // Allocating space for the destination node
  std::size_t dst_index = out.size();
  out.emplace_back();

  if (top->kind == add_v) {
    auto const &typed_top =
        static_cast<add_t const &>(*top);

    // Serializing left and right subtrees,
    // initializing the new node
    out[dst_index] = {flat_add_t{
        .left = serialize_impl(typed_top.left, out),
        .right =
            serialize_impl(typed_top.right, out)}};
  }

  if (top->kind == constant_v) {
    auto const &typed_top =
        static_cast<constant_t const &>(*top);
    out[dst_index] = {
        flat_constant_t{.value = typed_top.value}};
  }

  return dst_index;
}

constexpr std::vector<flat_node_t>
serialize(tree_ptr_t const &tree) {
  std::vector<flat_node_t> result;
  serialize_impl(tree, result);
  return result;
}
\end{lstlisting}

Listing \ref{lst:flat_backend_serialization} shows the implementation of the
serialization step. Note that all the code that was presented so far
is ordinary \cpp code, except for the functions being \gls{constexpr}.
From there we can implement the last transformation, which consists
in transferring the serialized data into a static array.

\begin{lstlisting}[
  language=C++,
  label=lst:serialize_as_array,
  caption=Definition of \lstinline{serialize_as_array}
]{}
template <tree_generator auto Fun>
constexpr auto serialize_as_array() {
  constexpr std::size_t Size = serialize(Fun()).size();
  std::array<flat_node_t, Size> result;
  std::ranges::copy(serialize(Fun()), result.begin());
  return result;
}
\end{lstlisting}

Listing \ref{lst:serialize_as_array} shows the implementation of a helper
function that takes a \gls{constexpr} tree generator function as an input,
serializes the result, and returns it as a static array.
This can be done because the generator function is \gls{constexpr}, therefore it
can be used to produce \gls{constexpr} values. The size of the resulting
\lstinline{std::vector} can be stored at compile time to set the size
of a static array.
The elements contained in the resulting array are \glspl{litval} since they do
not hold pointers to dynamic memory, therefore the array itself is a
\gls{litval} as well.

To complete the implementation, we must implement a code generation function
that accepts a serialized tree as an input as shown in listing
\ref{lst:flat_codegen}. Note that this function is almost identical to
the one shown in listing \ref{lst:pass-by-generator-example}.
The major difference is that \lstinline{TreeGenerator} is called only
twice regardless of the size of the tree. This allows much better scaling
as we will see in \ref{lbl:compile-time-eval}.
The downside is that it requires the implementation of an ad hoc data
structure and a serialization function, which might be more or less complex
depending on the complexity of the original tree structure.

\subsection{
  Using algorithms with serialized outputs
}
\label{lbl:codegen-from-rpn}

\begin{figure}[h]
\begin{tabular}{|l|l|l|}
\hline
Current token & Action & Stack \\
\hline
\lstinline|2| & Stack \lstinline|2| & \lstinline|2| \\
\lstinline|3| & Stack \lstinline|3| & \lstinline|2|, \lstinline|3| \\
\lstinline|2| & Stack \lstinline|2| & \lstinline|2|, \lstinline|3|, \lstinline|2| \\
\lstinline|*| & Multiply \lstinline|2| and \lstinline|3| & \lstinline|2|, \lstinline|6| \\
\lstinline|+| & Add \lstinline|2| and \lstinline|6| & \lstinline|8| \\
\hline
\end{tabular}
\centering
\caption{\gls{rpn} formula reading example}
\label{fig:rpn-reading-example}
\end{figure}

Parsing algorithms may output serialized data. In this case, the serialization
step described in \ref{lbl:flat-technique} is not needed, and the result
can be converted into a static array.
This makes the code generation process rather straightforward as no complicated
transformation is needed, while still scaling decently as we will see in
\ref{lbl:compile-time-eval} where we will be using a
Shunting Yard parser \cite{shunting-yard} to parse math formulae to a
\gls{rpn}, which is its postfix notation.

Once converted into its postfix notation,
a formula can be read using the following method:
reading symbols in order,
putting constants and variables on the top of a stack, and
when a function $f$ or operator of arity $N$ is being read,
unstacking $N$ values and stack the result of $f$ applied
to the $N$ operands.

Figure \ref{fig:rpn-reading-example} shows a formula reading example with
the formula \lstinline{2 + 3 * 2}, or \lstinline{2 3 2 * +} in
reverse polish notation. Starting from there,
we will see how code can be generated using RPN representations
of addition trees in \cpp.

\begin{lstlisting}[
  language=C++,
  label=lst:flat_codegen,
  caption=Code generation implementation for the flat backend
]{}
template <auto const &Tree, std::size_t Index>
constexpr auto codegen_aux() {
  static_assert(Index != null_index,
                "Ill-formed tree (null index)");

  if constexpr (std::holds_alternative<flat_add_t>(
                    Tree[Index])) {
    constexpr auto top =
        std::get<flat_add_t>(Tree[Index]);

    // Recursive code generation for left and right
    // children
    auto eval_left = codegen_aux<Tree, top.left>();
    auto eval_right = codegen_aux<Tree, top.right>();

    // Code generation for current node
    return [=]() { return eval_left() + eval_right(); };
  }

  else if constexpr (std::holds_alternative<
                         flat_constant_t>(Tree[Index])) {
    constexpr auto top =
        std::get<flat_constant_t>(Tree[Index]);
    constexpr int Value = top.value;
    return []() { return Value; };
  }
}

/// Stores serialized representations
/// of tree generators' results.
template <tree_generator auto Fun>
static constexpr auto tree_as_array =
    serialize_as_array<Fun>();

template <tree_generator auto Fun>
constexpr auto codegen() {
  return codegen_aux<tree_as_array<Fun>, 0>();
}
\end{lstlisting}

\clearpage%

In listing \ref{lst:rpn_base_definitions}, we have the type definitions for an
RPN representation of an addition tree as well as \lstinline{gen_rpn_tree} which
returns an RPN equivalent of \lstinline{gen_tree}'s result.

Similar to the flat backend, an \lstinline{eval_as_array} takes care
of evaluating the \lstinline{std::vector} result into a statically
allocated array.

\begin{lstlisting}[
  language=C++,
  label=lst:rpn_base_definitions,
  caption=RPN example base type and function definitions
]{}
/// Type for RPN representation of a constant
struct rpn_constant_t {
  int value;
};

/// Type for RPN representation of an addition
struct rpn_add_t {};

/// Type for RPN representation of an arbitrary symbol
using rpn_node_t =
    std::variant<rpn_constant_t, rpn_add_t>;

/// RPN equivalent of gen_tree
constexpr std::vector<rpn_node_t> gen_rpn_tree() {
  return {rpn_constant_t{1}, rpn_constant_t{2},
          rpn_add_t{}, rpn_constant_t{3}, rpn_add_t{}};
}
\end{lstlisting}

In listing \ref{lst:rpn_codegen}, we have function definitions for the
implementation of \lstinline{codegen} which takes an RPN generator function,
and generates a function that evaluates the tree.

The code generation process happens by updating an operand stack represented
by a \lstinline{kumi::tuple}. It is a standard-like tuple type with additional
element access, extraction, and modification functions.
We are using it to store functions that evaluate parts of the subtree.

The algorithm reads \textbf{symbols} (\ie functions, operators, constants, and
variables) one by one from a \textbf{symbol queue}, and generates
\textbf{operands} (\ie \cpp lambdas that return a result) that are stored on an
\textbf{operand stack}. For each \textbf{symbol} being read from the queue:

\begin{itemize}
\item If a \textbf{constant} or \textbf{variable} is read, a lambda that
evaluates its value is added to the operand stack.
Since the operands are \cpp lambdas, they can return
anything users may want them to return, such as a \cpp variable
or even another function's result.

\item If a \textbf{function} or \textbf{operator} of arity $N$ is read,
$N$ operands will be consumed from the top of the \textit{operand stack}
to generate a new one. This new operand evaluates the consumed operands passed
to the function corresponding to the symbol being read from the queue.
\end{itemize}

Once all symbols are read, there should be only one \textbf{operand} remaining
on the operand stack: a function that evaluates the whole formula.

\begin{lstlisting}[
  language=C++,
  caption=Codegen implementation for RPN formulas,
  label=lst:rpn_codegen
]{}
template <auto const &RPNTree, std::size_t Index = 0>
constexpr auto
codegen_impl(kumi::product_type auto stack) {
  // The end result is the last top stack operand
  if constexpr (Index == RPNTree.size()) {
    static_assert(stack.size() == 1, "Invalid tree");
    return kumi::back(stack);
  }

  else if constexpr (std::holds_alternative<
                         rpn_constant_t>(
                         RPNTree[Index])) {
    // Append the constant operand
    auto new_operand = [=]() {
      return get<rpn_constant_t>(RPNTree[Index]).value;
    };
    return codegen_impl<RPNTree, Index + 1>(
        kumi::push_back(stack, new_operand));
  }

  else if constexpr (std::holds_alternative<rpn_add_t>(
                         RPNTree[Index])) {
    // Fetching 2 top elements
    // and popping them off the stack
    auto left = kumi::get<stack.size() - 1>(stack);
    auto right = kumi::get<stack.size() - 2>(stack);
    auto stack_remainder =
        kumi::pop_back(kumi::pop_back(stack));

    // Append new operand and process next element
    auto new_operand = [=]() { return left() + right(); };
    return codegen_impl<RPNTree, Index + 1>(
        kumi::push_back(stack_remainder, new_operand));
  }
}

template <auto Fun>
static constexpr auto rpn_as_array = eval_as_array<Fun>();

template <auto Fun> constexpr auto codegen() {
  return codegen_impl<rpn_as_array<Fun>, 0>(
      kumi::tuple{});
}
\end{lstlisting}

\clearpage%

\subsection{
  Conclusion
}

We now have a set of methods that allows us to convert \gls{constexpr}
\glspl{ast} into \cpp code, each with their flaws and advantages with regard
to compilation times, implementation difficulty, or simply final code size.
So far, all of them require advanced \cpp knowledge and tinkering to get
Clang to compile them. But as difficult as their implementation may be,
it is still possible to design toolchains that minimize the amount of code
related to \gls{ast} transformation, and keep most of the computing in
\gls{constexpr} functions that are easier to write and maintain
for most \cpp developers.

We can alreaady notice a few differences between those techniques:
\gls{pbg}-based backends seem to have much higher compilation times,
and they induce errors that are harder to diagnose despite the implementation
being seemingly straightforward.
The Flat backend however, requires more work to serialize the \gls{ast},
but serializing a pointer tree does not require much advanced \cpp knowledge.
Moreover, the serialization code can be executed, debugged, and tested in a
regular run-time environment.

So far, it seems that the \gls{pbg} method is more suitable for prototyping,
whereas the serialization method is a better fit for larger meta-programs that
might benefit from having better readability and performance at scale.

\end{document}
