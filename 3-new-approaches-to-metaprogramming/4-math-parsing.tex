\documentclass[../main]{subfiles}
\begin{document}

\section{
  Math parsing and high performance code generation
}

Now that we have a good overview of what to

\subsection{
  The Shunting-Yard algorithm
}

The Shunting Yard algorithm \cite{shunting-yard} is a simple slgorithm
that transforms mathematical formulas from common notation (\ie infix notation)
to a postfix notation, also known as the \gls{rpn}. It is a representation where
operators are put not between but after operands, for example the following
formula \lstinline{1 + (2 * 3)} becomes \lstinline{1 2 3 * +}.

Postfix notation is known for being trivially parsable.
The parsing algorithm consists in reading a formula symbol by symbol.
If the symbol is a constant, it is pushed on the top of a stack.
If the symbol is an operator, the constants on top of the stack are consumed
by the operation and replaced by the result.

\begin{figure}
\centering
\begin{tabular}{| c | l | l |}
\hline
Token & Operation         & Stack \\
\hline
1     & Stack 1           & 1 \\
2     & Stack 2           & 1, 2 \\
3     & Stack 3           & 1, 2, 3 \\
*     & Multiply 2 and 3  & 1, 6 \\
+     & Add 1 and 6       & 7 \\
\hline
\end{tabular}
\caption{\gls{rpn} parsing sequence example}
\label{fig:rpn-parsing-example}
\end{figure}

Figure \ref{fig:rpn-reading-example}

\subsection{
  Code generation from a postfix math notation
}

\subsection{
  Using Blaze for high performance code generation
}

% NOTE: refer to previous chapters for Blaze introduction and examples

\subsection{
  Conclusion: a complete toolchain for \acrlong{hpc} code generation from math formulas
}

\end{document}
