\documentclass[../main]{subfiles}
\begin{document}

\section{
  Advanced constexpr programming
}
\label{lbl:constexpr-programming}

% TODO: dig this for references https://youtu.be/q6X9bKpAmnI

As mentioned in section \ref{lbl:cpp-meta-constructs}, the \gls{constexpr}
allows variables and functions to be used in \gls{consteval},
making a whole subset of the \cpp language itself usable for compile-time logic.

As the language evolves, this subset becomes larger as more \cpp become
usable in \gls{constexpr} functions. Some of the most notable evolutions
are allowing \gls{constexpr} memory allocations and virtual function calls
in \glspl{consteval} since \cpp20.
These two additions make dynamic memory and heritage-based polymorphism
possible in \gls{constexpr} functions. Therefore more regular \cpp code can be
executed at compile time, including parsers for arbitrary languages.

This section will cover advanced technicalities of \gls{constexpr} programming,
including \gls{constexpr} memory allocation and the restrictions that make
\gls{constexpr} allocated memory difficult to use for \cpp code generation.

\clearpage%

\subsection{
  \gls{constexpr} functions and \gls{constexpr} variables
}

\textbf{Functions} that are \gls{constexpr} qualified can still be used
in other contexts than \gls{consteval} happening at compile time.
In non-\gls{consteval}, \gls{constexpr} functions can still call
non-\gls{constexpr} functions. But in \glspl{consteval}, \gls{constexpr}
functions must only call other \gls{constexpr} functions.
This applies to methods as well. In order to make a \cpp class or structure
fully usable in \glspl{consteval}, its methods --including the constructors
and destructor-- must be \gls{constexpr}.

\textbf{Variables} that are \gls{constexpr} qualified can be
used in constant expressions. Note that they are different from
non-\gls{constexpr} variables used in \gls{constexpr} functions.
There are more requirements on \gls{constexpr} variables.
Their values must be literal, meaning that memory allocated in \gls{constexpr}
function bodies cannot be stored in \gls{constexpr} variables.

\subsection{
  \gls{constexpr} memory allocation and \gls{constexpr} virtual function calls
}

With \cpp20, \lstinline{std::allocate} and \lstinline{std::deallocate}
became \gls{constexpr} functions, allowing memory allocations to happen in
\glspl{consteval}.

However \gls{constexpr} allocated memory is not transient, \ie memory allocated in
constant expressions cannot be stored in \gls{constexpr} variables, and \glspl{nttp}
cannot hold pointers to \gls{constexpr} allocated memory either.

Note that this restriction does not mean that data stored in \gls{constexpr} memory
cannot be passed through. There are techniques to use data in \gls{constexpr}
allocated memory

\begin{lstlisting}[
  language=C++,
  caption=Illustration of constraints on \gls{constexpr} allocated memory,
  label=lst:cx-examples
]{}
// Constexpr function generate returns a non-literal value
constexpr std::vector<int> generate() { return {1,2,3,4,5}; }

// Function template foo takes a polymorphic NTTP
template<auto bar> constexpr int foo() { return 1; }

// generate's return value cannot be stored
// in a constexpr variable or used as a NTTP,
// but it can be used to produce other literals

// constexpr auto a = generate();         // ERROR
constexpr auto b = generate().size();     // OK
// constexpr auto c = foo<generate()>();  // ERROR
constexpr auto d = foo<&generate>();      // OK
\end{lstlisting}

\clearpage%

Let's have a closer look at the four assignment cases:

\begin{itemize}
\item Case \lstinline{a}: \lstinline{generate}'s return value is non-literal
      and therefore cannot be stored in a \gls{constexpr} variable.
\item Case \lstinline{b}: \lstinline{generate}'s return value is used in a
      constant expression to produce a literal value.
      Therefore the expression's result can be stored in a \gls{constexpr} variable.
\item Case \lstinline{c}: similarly to case \lstinline{a},
      \lstinline{generate}'s return value cannot be used as an \gls{nttp} because it
      is not a literal value.
\item Case \lstinline{d}: function references are allowed as \glspl{nttp}.
\end{itemize}

Notice how the last example works around restrictions of \gls{constexpr} allocations
by using a generator function instead of passing the non-empty
\lstinline{std::vector<int>} value directly. This technique along with the
definition of lambdas can be used to explore more complex structures returned by
\gls{constexpr} functions such as pointer trees.

Moreover, \gls{constexpr} allocated memory being non transient does not mean that its
content cannot be transferred to \gls{nttp} compatible data structures.

\begin{lstlisting}[
  language=C++,
  caption=Transformation from a dynamic vector to a constexpr static array,
  label=lst:cx-vector-to-array
]{}
constexpr auto generate_as_array() {
  constexpr std::size_t array_size =
      generate().size();
  std::array<int, array_size> res{};
  std::ranges::copy(generate(), res.begin());
  return res;
}

constexpr auto e = generate_as_array();        // OK
constexpr auto f = foo<generate_as_array()>(); // OK
\end{lstlisting}

Listing \ref{lst:cx-vector-to-array} shows how \lstinline{generate}'s result
can be evaluated into a static array. Static arrays are literal as long
as the values they hold are literal. Therefore the result of
\lstinline{generate_as_array} can be stored in a \gls{constexpr} variable or
used directly as an \gls{nttp} for code generation.

The addition of \gls{constexpr} memory allocation goes hand in hand
with the ability to use virtual functions in \glspl{consteval}.
This feature allows calls to virtual functions in constant expressions
\cite{virtual-constexpr}. This allows inheritance-based polymorphism in
\gls{constexpr} programming when used with \gls{constexpr} allocation of
polymorphic types.

% \section{C++23 experiments for implementing EDSLs of arbitrary syntaxes}

% We explore two cases of \gls{constexpr} parser implementations for small languages
% that return \glspl{ast} using \gls{constexpr} allocated memory through
% standard containers, and backends that transform the \gls{constexpr} function results
% into programs through templates using different techniques. The use cases we
% consider are:
%
% \begin{itemize}
%   \item The Brainfuck as a Turing machine emulation language. Its main interest
%         is its large corpus of example programs.
%   \item A basic math language supporting infix operators and function calls.
% \end{itemize}

% The Poacher project was started to experiment ways to leverage \gls{constexpr}
% programming for the integration of embedded domain specific languages of
% arbitrary syntaxes. Considering the current restrictions on \gls{constexpr} allocated
% memory, going from a static string representation of a program, mathematical formula, or
% regular expression to a compiled program using \gls{constexpr} programming might not
% be trivial.

% https://www.cs.rhul.ac.uk/research/languages/csle/lookahead_backtrack.html
% https://www.boost.org/doc/libs/1_81_0/libs/spirit/doc/html/spirit/references.html
% https://github.com/antlr/antlr4

% TODO: Expose the issue clearly!!

% https://martinfowler.com/books/dsl.html

\end{document}
